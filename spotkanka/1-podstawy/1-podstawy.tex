\documentclass{article}
%
\usepackage{multicol}
\usepackage{polski}
\usepackage[utf8]{inputenc}
\usepackage{bridge_macros}
\setlength{\textwidth}{7in}
\setlength{\textheight}{9in}
\setlength{\oddsidemargin}{-.4in}
\setlength{\topmargin}{-.5in}
%
\begin{document}
\section{Podstawy}
52 karty, czterech graczy siedzących na pozycjach \textbf{N}orth \textbf{S}outh i \textbf{E}ast i \textbf{W}est.
Gracze \textbf{NS} i \textbf{EW} grają w parach. Rozdanie zaczynamy od \textbf{licytacji} w celu ustalenia
rozgrywającego, wysokości kontraktu (czyli tego, ile \textbf{lew} należy wziąć) oraz koloru atutowego. Osoba po lewej
stronie rozgrywającego oddaje \textbf{wist}, czyli zagrywa jakąś kartę. Partner rozgrywającego staje się 
\textbf{dziadkiem} - wykłada swoje karty na stół i od teraz rozgrywający dysponuje jego kartami.
\section{Licytacja}
Do dyspozycji mamy 4 kolory od najmłodszego (\c\d\h\s) i piąty, najstarszy, tak zwane \textbf{bez atu}. Łącząc kolor z wysokością (1-7)
otrzymujemy odzywkę (np. 3\s). Dodatkowo możemy zalicytować: \textbf{PAS}, \textbf{x} (kontra) i \textbf{xx} (rekontra).
Kontrę można dać tylko na odzywkę przeciwnika, rekontrę tylko na kontrę. Zalicytować można odzywkę, która jest
\textbf{wyżej} od poprzedniej tj. jeśli ostatnią zalicytowaną odzywką będzie 3\h, to można zalicytować 3\s, 3BA, 4\c,
4\d\ itd. Licytacja toczy się do momentu, w którym po jakiejś odzywce nastąpią 3 pasy, a sama odzywka staje się
\textbf{kontraktem}. Rozgrywającym jest ten, który jako pierwszy zalicytował kolor kontraktu i musi wziąć 6 + wysokość
kontraktu lew, a kolor kontraktu staje się tak zwanym \textbf{atu}.\\
Wyróżnia się 4 rodzaje kontraktów:
\begin{itemize}
	\item \textbf{Częściówki} - kontrakty poniżej końcówki.
	\item \textbf{Końcówki} - 3BA, 4 w kolor \textbf{starszy} (\h\s)\ lub 5 w kolor \textbf{młodszy} (\c\d).
	\item \textbf{Szlemiki} - kontrakty na poziomie 6. 
	\item \textbf{Szlemy} - kontrakty na poziomie 7. 
\end{itemize}
\section{Rozgrywka}
Obrońca po lewej stronie rozgrywającego zagrywa dowolną kartą (tzw. \textbf{wist}), wykłada się dziadek, każdy po kolei
(pamiętajmy, że to rozgrywający zagrywa karty dziadka) dokłada do koloru. Osoba,
która da najwyższą kartę bierze \textbf{lewę} i zagrywa następną kartę. Jeśli nie mamy do koloru, możemy zrzucić
dowolną kartę, ale (o ile nie jest to kolor atutowy nie weźmiemy lewy. Atu jest on silniejsze od pozostałych kolorów. Wzięta lewa
liczy się dla całej pary, a nie jednego obrońcy. Po 13 lewach podsumowuje się wynik i przechodzi do następnego rozdania.

\end{document}
