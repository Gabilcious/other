\documentclass{article}
%
\usepackage{bridge_macros}
\newcommand\tab[1][1cm]{\hspace*{#1}}

\begin{document}
\noindent
Zad 1\\Grasz 3BA. Ile masz lew w każdym kolorze? Gdzie zamierzasz wziąć brakujące?
\begin{quote}
\decl{W wistuje \d D}{\hand{KD2}{AD}{A432}{K432}}{\hand{AW10}{5432}{K5}{A765}}
\end{quote}
Zad 2\\Ile lew przeciwnicy mogą wziąć od razu przy założeniu, że wistuje E? A ile, jeśli wistuje W?\\\\
\begin{tabular}{lp{2cm}l}
& \multicolumn{1}{c}{N} & \multicolumn{1}{c}{S}\\\\
1) & \s KW & \s 1032\\
2) & \s W  & \s K92\\
3) & \s 10  & \s D832\\
4) & \s W1087  & \s 2\\
\end{tabular}\\\\
Zad 3* (dla ambitnych)\\Teraz zakładamy, że przeciwnicy będą wychodzić dwa razy (tj. sytuacja, w której w dalszej
częsci któryś obrońca dojdzie ponownie). Dla każdego przykładu oczekujemy czterech wyników, w każdym z dwóch wyjść wychodzi E lub W.
Do tego zakładamy, że obrońca W ma zawsze nie mniej kart niż jego partner.\\\\
\begin{tabular}{lp{2cm}l}
& \multicolumn{1}{c}{N} & \multicolumn{1}{c}{S}\\\\
1) & \s KD10 & \s 32\\
2) & \s KD9  & \s 32\\
3) & \s D32  & \s W54\\
4) & \s D2  & \s W987\\
5) & \s AW2  & \s 543\\
\end{tabular}

\end{document}
% \noindent\rule{8cm}{0.4pt}
%
