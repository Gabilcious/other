\documentclass{article}
\usepackage{bridge_macros}

\begin{document}
\section*{Rozgrywa kontraktów bezatutowych}
Rogrywka w BA i w kolor trochę się różni, dlatego rozbijamy ten temat na dwa osobne. We wszystkich problemach poniżej
grane jest 3BA
\subsection*{Na co należy zwrócić uwagę?}
\subsubsection*{1. Po zobaczeniu dziadka przechodzimy do policzenia pewnych lew}.
% \begin{multicols*}{2}
\begin{quote}
\decl{W wistuje \s W}{\hand{K3}{KDW109}{W93}{A3}} {\hand{A2}{64}{AD102}{KDW5}}
\end{quote}
Liczymy:
\begin{itemize}
	\item[--] W pikach dwie (na A i K).
	\item[--] W kierach zero (niemniej możemy łatwo sobie wyrobić cztery lewy, oddając przeciwnikom na asa).
	\item[--] W karach jedna (po oddaniu na króla trzy)
	\item[--] W treflach cztery (AKDW).
\end{itemize}
Pewnych 7, dodatkowe 4 zdobywamy oddając na \h A i biorąc pozostałe kiery.
\subsubsection*{2. Musimy uważać na nasze słabe kolory}
Spójrzmy na bardzo podobne rozdanie do poprzedniego.
\begin{quote}
\decl{W wistuje \s W}{\hand{43}{KDW109}{W93}{A3}}{\hand{A2}{64}{AD102}{KDW5}}
\end{quote}
Teraz w pikach mamy tylko asa, co sprawia, że jeśli przeciwnicy coś wezmą, ściągną nam co najmniej 4 lewy w tym kolorze
(mają w sumie 9 kart, więc któryś z obrońców ma 5, my weźmiemy jedną na asa, on pozostałe 4).\\
Mamy 6 pewnych lew, znów moglibyśmy oddać na \h A, ale zgodnie z naszą analizą, przeciwnicy wezmą wtedy 4 piki i \h A obkładając nas bez jednej.
\textbf{Wniosek?} Musimy wziąć 9 lew zanim oddamy cokolwiek przeciwnikom. Szansę na to daje \textbf{impas} karowy!
Jeśli \d K znajduje się u \textbf{E}, to grając karo ze stołu i dokładając z ręki np. \d D, weźmiemy lewę. Jeśli
wykonamy impas trzykrotnie, weźmiemy dodatkowe 3 lewy karowe, co pozwoli nam zrealizować kontrakt.
\subsubsection*{3. Dbamy o komunikację}.
Wzięliśmy lewę \s A, przechodzimy na stół \c A i gramy \d 3 dokładając z ręki 10. Udało się, impas stał i \d 10 wzięła.
Chcemy powtórzyć impas ale... jesteśmy w ręce i okazuje się, że nie ma dojścia do stołu. Musimy zagrać tak, aby po
udanym impasie \textbf{zostać w stole}. Prosimy przeciwników o cofnięcie ostatniej lewy i gramy tym razem \d 9!
\textbf{puszczając ją wkoło} tj. dokładając coś niższego z ręki tak, aby 9 wzięła lewę (oczywiście jeśli zobaczymy
K bijemy go A).
\subsubsection*{Pytanie bonusowe o dodatnią część dywizji}
Dlaczego zagranie \d W zamiast \d 9 jest błędem?
\subsubsection*{4. Uważamy na niebezpiecznego przeciwnika}.
\begin{quote}
\decl{W wistuje \h 8}{\hand{72}{AK2}{ADW109}{432}}{\hand{K5}{D1065}{62}{ADW109}}
\end{quote}
Nasze lewy: 0 w \s, 4 w \h\ (po tym wiście, jeśli dołożymy z dziadka blotkę, to weźmiemy lewy na D i 10), 1 w \d\ 
i 1 w \c. Potrzebujemy 3 dodatkowych lew, możemy je wziąć zarówno w karach jak i w treflach, wykonując impas. Naszym
słabym kolorem są piki. Co się stanie, jeśli E weźmie lewę i zagra w pika? Jeśli A\s\ jest źle położony (za K)
przeciwnicy wezmą dużo pików. W natomiast nie może sam zagrać w pika, bo wtedy nasz król nie pozwoli przeciwnikom
zebrać lew w tym kolorze. E jest w takim razie \textbf{niebezpiecznym przeciwnikiem} i nie możemy dopuścić go do ręki.
\textbf{Wniosek?} Gramy na impas trefl, bo nawet jeśli się nie uda, do ręki dojdzie bezpieczny obrońca a my wyrobimy
sobie 3 dodatkowe lewy, wygrywając kontrakt.
\end{document}
% \noindent\rule{8cm}{0.4pt}
%
