\documentclass{article}

\usepackage{polski}
\usepackage[utf8]{inputenc}
\usepackage{amssymb}
\usepackage{geometry}

\newcommand{\pluseq}{\mathrel{+}=}

\title{Złożoność obliczeniowa 1.1}
\author{Krystian Dowolski}

\begin{document}
	\maketitle
    Niech MT będzie standardową maszyną Turinga akceptującą dany język. Oznaczmy przez Q jej stany, i przez $\delta$
    jej funkcję przejścia i przez k maksymalną długość słowa zapisanego na taśmie. Tworzymy maszynę MMT
    $$ Q' = \{(q, i, j, nextRound): q\in Q, i,j\in \mathbb{N}, i,j\leq k, nextRound\in \{0,1\}\}$$
    j oznacza aktualną pozycję głowicy (licząc od pierwszego symbolu), i - ile musimy jeszcze się przesunąć, żeby
    głowica była w odpowiednim miejscu. nextRound natomiast jest flagą mówiącą, czy zaczęliśmy kolejne przejście od
    początku taśmy.\\
    Idea jest taka, że gdy MT wykonuje przejście ze stanu $q_1$ (i symbolu a) do $q_2$ (wpisując symbol b) i przesuwa głowicę,
    my wchodzimy w stan, którego zadaniem jest poczekać, aż MMT ustawi głowicę w miejscu, w którym powinna być dla MT. Dlatego najpierw
    dodajemy
    $$\delta' = ((q_1, 0, j, 1), a) = ((q_2, j+\{-1,0,1\}^*, 0, 0), b)$$
    gdzie * to -1 dla przesunięcia w lewo, 0 gdy brak przesunięcia i 1 dla przesunięcia w prawo.\\
    Gdy nextRound = 0, zwyczajnie przepychamy głowicę dalej bez zmian, dopóki nie natrafimy na $\bot$ tj.
    $$\delta' \pluseq((q, i, j, 1), a) = ((q, i, j, 1, a)), a\in \Gamma \setminus \bot $$
    $$\delta' \pluseq ((q, i, 0, 1), \bot) = ((q, i-1, 1, 0), \bot), i > 0$$
    Kiedy nextRound jest ustawione na 1, zbijamy i w każdym ruchu:
    $$\delta' \pluseq ((q, i, j, 0), a) = ((q, i-1, j+1, 0), a), i > 0 $$

\end{document}
