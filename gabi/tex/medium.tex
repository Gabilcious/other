%% start of file `template.tex'.
%% Copyright 2006-2013 Xavier Danaux (xdanaux@gmail.com).
%
% This work may be distributed and/or modified under the
% conditions of the LaTeX Project Public License version 1.3c,
% available at http://www.latex-project.org/lppl/.


\documentclass[11pt,a4paper,sans]{moderncv}        % possible options include font size ('10pt', '11pt' and '12pt'), paper size ('a4paper', 'letterpaper', 'a5paper', 'legalpaper', 'executivepaper' and 'landscape') and font family ('sans' and 'roman')
\usepackage{enumitem}
\setlist{leftmargin=15.5mm}
\usepackage{footmisc} % enabling footnotes.


% moderncv themes
\moderncvstyle{classic}                             % style options are 'casual' (default), 'classic', 'oldstyle' and 'banking'
\moderncvcolor{green}                               % color options 'blue' (default), 'orange', 'green', 'red', 'purple', 'grey' and 'black'
%\renewcommand{\familydefault}{\sfdefault}         % to set the default font; use '\sfdefault' for the default sans serif font, '\rmdefault' for the default roman one, or any tex font name
%\nopagenumbers{}                                  % uncomment to suppress automatic page numbering for CVs longer than one page

% character encoding
\usepackage[utf8]{inputenc}                       % if you are not using xelatex ou lualatex, replace by the encoding you are using
%\usepackage{CJKutf8}                              % if you need to use CJK to typeset your resume in Chinese, Japanese or Korean

% adjust the page margins
\usepackage[scale=0.75]{geometry}
%\setlength{\hintscolumnwidth}{3cm}                % if you want to change the width of the column with the dates
%\setlength{\makecvtitlenamewidth}{10cm}           % for the 'classic' style, if you want to force the width allocated to your name and avoid line breaks. be careful though, the length is normally calculated to avoid any overlap with your personal info; use this at your own typographical risks...

% personal data
\name{Opętane}{Medium}
\title{}                               % optional, remove / comment the line if not wanted
\photo[64pt][0.4pt]{medium.jpg}                       % optional, remove / comment the line if not wanted; '64pt' is the height the picture must be resized to, 0.4pt is the thickness of the frame around it (put it to 0pt for no frame) and 'picture' is the name of the picture file

% to show numerical labels in the bibliography (default is to show no labels); only useful if you make citations in your resume
%\makeatletter
%\renewcommand*{\bibliographyitemlabel}{\@biblabel{\arabic{enumiv}}}
%\makeatother
%\renewcommand*{\bibliographyitemlabel}{[\arabic{enumiv}]}% CONSIDER REPLACING THE ABOVE BY THIS

% bibliography with mutiple entries
%\usepackage{multibib}
%\newcites{book,misc}{{Books},{Others}}
%----------------------------------------------------------------------------------
%            content
%----------------------------------------------------------------------------------
\begin{document}
%\begin{CJK*}{UTF8}{gbsn}                          % to typeset your resume in Chinese using CJK
%-----       resume       ---------------------------------------------------------
\makecvtitle
Dziewczyna, która wiodła normalne życie do czasu, gdy opętały ją złe moce. Zaczęła wtedy słyszeć dziwne głosy
w głowie oraz zauważać otaczające ją duchy, demony i inne paranormalne istoty. Z każdym dniem boi się coraz bardziej
tego, co ją może czekać. Czuje także, że coś próbuje ją opętać i stara się wszelkimi sposobami bronić przed tym.

\section{Wygląd}
\begin{itemize}
	\item Potargane włosy
	\item Przygarbiona
	\item Blade, przestraszone lico
	\item Podkrążone oczy
\end{itemize}

\section{Wyposażenie}
\begin{itemize}
	\item Różnorakie amulety
	\item Wahadełko do kontaktu z duchami
	\item Długa, biała koszula
\end{itemize}


\section{Zadania}
1pkt:
\begin{itemize}
	\item Rozdać kilkunastu znajomym amulety ochronne\footnote{Oczywiście z wyjaśnieniem, czemu służą.}
	\item Kilka razy zacząć patrzeć się w jeden punkt i zapytać ze strachem ludzi w pobliżu\footnote{Co najmniej
		jednego niebędącego Tobą.}, czy też to widzieli
	\item Przez pół godziny prowadzić werbalny dialog z paranormalnymi istotami\footnote{5 świadków.}
	\item Poprosić kogoś, aby wziął od Ciebie telefon\footnote{Na cały tydzień!} i pod żadnym pozorem Ci nie
		oddawał\footnote{Musi sie zgodzić.}
\end{itemize}
\newpage

2pkt:
\begin{itemize}
	\item Opdrawić rytuał ochronny razem z bliskim\footnote{Za użycie krwi owcy dodatkowy punkt.}
	\item Przez całe zajęcia stać z tyłu w kącie argumentując, że boisz się do 'tego'\footnote{Bynajmniej chodzi tutaj
			o problemy demonstrowane na tablicy.} podejść
	\item Wziąć w panice czyjś zeszyt i wyrzucić przez okno
	\item Zaprzyjaźnić się z kimś opętanym
\end{itemize}

5pkt:
\begin{itemize}
	\item Pójść do parafii i poprosić księdza o egzorcystę\footnote{Odprawienie egzorcyzmu należy do zakresu.}
	\item Zacząć zwijać się w konwulsjach tocząc pianę i wrzeszcząc, żeby Cię wreszcie zostawił\footnote{10+
		obserwatorów.}
\end{itemize}

10pkt:
\begin{itemize}
	\item Przeprowadzić próbę samobójczą\footnote{niegroźną, acz wiarygodną.}
\end{itemize}
\end{document}


%% end of file `template.tex'.
