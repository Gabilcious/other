%% start of file `template.tex'.
%% Copyright 2006-2013 Xavier Danaux (xdanaux@gmail.com).
%
% This work may be distributed and/or modified under the
% conditions of the LaTeX Project Public License version 1.3c,
% available at http://www.latex-project.org/lppl/.


\documentclass[11pt,a4paper,sans]{moderncv}        % possible options include font size ('10pt', '11pt' and '12pt'), paper size ('a4paper', 'letterpaper', 'a5paper', 'legalpaper', 'executivepaper' and 'landscape') and font family ('sans' and 'roman')
\usepackage{enumitem}
\setlist{leftmargin=15.5mm}
\usepackage{footmisc} % enabling footnotes.


% moderncv themes
\moderncvstyle{classic}                             % style options are 'casual' (default), 'classic', 'oldstyle' and 'banking'
\moderncvcolor{green}                               % color options 'blue' (default), 'orange', 'green', 'red', 'purple', 'grey' and 'black'
%\renewcommand{\familydefault}{\sfdefault}         % to set the default font; use '\sfdefault' for the default sans serif font, '\rmdefault' for the default roman one, or any tex font name
%\nopagenumbers{}                                  % uncomment to suppress automatic page numbering for CVs longer than one page

% character encoding
\usepackage[utf8]{inputenc}                       % if you are not using xelatex ou lualatex, replace by the encoding you are using
%\usepackage{CJKutf8}                              % if you need to use CJK to typeset your resume in Chinese, Japanese or Korean

% adjust the page margins
\usepackage[scale=0.75]{geometry}
%\setlength{\hintscolumnwidth}{3cm}                % if you want to change the width of the column with the dates
%\setlength{\makecvtitlenamewidth}{10cm}           % for the 'classic' style, if you want to force the width allocated to your name and avoid line breaks. be careful though, the length is normally calculated to avoid any overlap with your personal info; use this at your own typographical risks...

% personal data
\name{Korpo}{Szczur}
\title{Tak jest, szefie!}                               % optional, remove / comment the line if not wanted
\photo[64pt][0.4pt]{korpo.jpg}                       % optional, remove / comment the line if not wanted; '64pt' is the height the picture must be resized to, 0.4pt is the thickness of the frame around it (put it to 0pt for no frame) and 'picture' is the name of the picture file

% to show numerical labels in the bibliography (default is to show no labels); only useful if you make citations in your resume
%\makeatletter
%\renewcommand*{\bibliographyitemlabel}{\@biblabel{\arabic{enumiv}}}
%\makeatother
%\renewcommand*{\bibliographyitemlabel}{[\arabic{enumiv}]}% CONSIDER REPLACING THE ABOVE BY THIS

% bibliography with mutiple entries
%\usepackage{multibib}
%\newcites{book,misc}{{Books},{Others}}
%----------------------------------------------------------------------------------
%            content
%----------------------------------------------------------------------------------
\begin{document}
%\begin{CJK*}{UTF8}{gbsn}                          % to typeset your resume in Chinese using CJK
%-----       resume       ---------------------------------------------------------
\makecvtitle
Ambitna, sumienna i pracowita. Zorganizowana w każdym calu, świadoma tego, że decyzja szefostwa to świętość, pnie się
po kolejnych korpo-szczebelkach kariery i uważa to za jedyną słuszną drogę życiową. Zrobi wszystko, aby przypodobać
się osobom wyżej postawionym, ale chętnie też wesprze tych niżej, gdyż uważa, że każdy, brzemienną pracą, powinien
osiągnąć sukces.

\section{Wygląd}
\begin{itemize}
	\item Zawsze elegancka
	\item Wyprasowane ubrania
	\item Zadbane włosy i paznokcie
\end{itemize}

\section{Wyposażenie}
\begin{itemize}
	\item Ołówkowa spódnica
	\item Okrągłe okulary
	\item Ulubiony terminarzyk pod pachą
\end{itemize}

\section{Zadania}
1pkt:
\begin{itemize}
	\item Umówić się z kilkoma osobami sprawdzając dostępność, wpisując w terminarz i prosząc ich o podpis
	\item Przeprowadzić kilku osobom zaawansowany test pod tytułem 'do którego korpo nadajesz się najbardziej'
	\item Poznać nową osobę w równym stopniu zafascynowaną pracą w korpo
	\item Uczestniczyć we wszystkich planowych\footnote{Nie liczą się tygodnie w 20+\% wolne od nauki.} zajęciach na MIMie...
\end{itemize}
\newpage

2pkt:
\begin{itemize}
	\item ... i to zawsze być 5 minut przed czasem
	\item Przekonać znajomego do tego, by zaczął\footnote{Czyli że wcześniej nie korzystał.} korzystać z papierowego terminarza
	\item Stwierdzić, że ktoś zrobił źle swoje zadanie\footnote{Musiał poświęcić na nie co najmniej pół godziny.}, wrócić do
		pierwotnego stanu\footnote{Skreślić, usunąć plik, nabrudzić etc.} i powiedzieć 'tym razem Cię z tym
		przypilnuję'\footnote{Może je wtedy zrobić dokładnie tak samo.}
	\item Po teście z zadania za 1 punkt, stwierdzić, że do żadnego i przeprowadzić godzinne szkolenie, aby stan ten odmienić
\end{itemize}

5pkt:
\begin{itemize}
	\item Zrobić prezentację\footnote{30+ min.} dla 10+ osób na temat 'Dlaczego warto pracować w korpo'
	\item Dostać się szczebel wyżej w karierze korzystając z flirtu i seksapilu\footnote{Polecam sztuczkę 'położę sobie
		jego rękę na (nie byle jakim zresztą) udzie'.}
\end{itemize}

10pkt:
\begin{itemize}
	\item Sprawić, by ktoś zwolnił się z obecnej pracy i złożył CV do korporacji\footnote{Z zamiarem dostania się.}
\end{itemize}
\end{document}


%% end of file `template.tex'.
