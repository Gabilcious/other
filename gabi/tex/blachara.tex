%% start of file `template.tex'.
%% Copyright 2006-2013 Xavier Danaux (xdanaux@gmail.com).
%
% This work may be distributed and/or modified under the
% conditions of the LaTeX Project Public License version 1.3c,
% available at http://www.latex-project.org/lppl/.


\documentclass[11pt,a4paper,sans]{moderncv}        % possible options include font size ('10pt', '11pt' and '12pt'), paper size ('a4paper', 'letterpaper', 'a5paper', 'legalpaper', 'executivepaper' and 'landscape') and font family ('sans' and 'roman')
\usepackage{enumitem}
\setlist{leftmargin=15.5mm}
\usepackage{footmisc} % enabling footnotes.


% moderncv themes
\moderncvstyle{classic}                             % style options are 'casual' (default), 'classic', 'oldstyle' and 'banking'
\moderncvcolor{green}                               % color options 'blue' (default), 'orange', 'green', 'red', 'purple', 'grey' and 'black'
%\renewcommand{\familydefault}{\sfdefault}         % to set the default font; use '\sfdefault' for the default sans serif font, '\rmdefault' for the default roman one, or any tex font name
%\nopagenumbers{}                                  % uncomment to suppress automatic page numbering for CVs longer than one page

% character encoding
\usepackage[utf8]{inputenc}                       % if you are not using xelatex ou lualatex, replace by the encoding you are using
%\usepackage{CJKutf8}                              % if you need to use CJK to typeset your resume in Chinese, Japanese or Korean

% adjust the page margins
\usepackage[scale=0.75]{geometry}
%\setlength{\hintscolumnwidth}{3cm}                % if you want to change the width of the column with the dates
%\setlength{\makecvtitlenamewidth}{10cm}           % for the 'classic' style, if you want to force the width allocated to your name and avoid line breaks. be careful though, the length is normally calculated to avoid any overlap with your personal info; use this at your own typographical risks...

% personal data
\name{Typowa}{Blachara}
\title{Imprezy, chłopaki i inne takie takie...}                               % optional, remove / comment the line if not wanted
\photo[64pt][0.4pt]{blachara.jpg}                       % optional, remove / comment the line if not wanted; '64pt' is the height the picture must be resized to, 0.4pt is the thickness of the frame around it (put it to 0pt for no frame) and 'picture' is the name of the picture file

% to show numerical labels in the bibliography (default is to show no labels); only useful if you make citations in your resume
%\makeatletter
%\renewcommand*{\bibliographyitemlabel}{\@biblabel{\arabic{enumiv}}}
%\makeatother
%\renewcommand*{\bibliographyitemlabel}{[\arabic{enumiv}]}% CONSIDER REPLACING THE ABOVE BY THIS

% bibliography with mutiple entries
%\usepackage{multibib}
%\newcites{book,misc}{{Books},{Others}}
%----------------------------------------------------------------------------------
%            content
%----------------------------------------------------------------------------------
\begin{document}
%\begin{CJK*}{UTF8}{gbsn}                          % to typeset your resume in Chinese using CJK
%-----       resume       ---------------------------------------------------------
\makecvtitle
Dziewczyna, która leci na 'blachę'. Jedyną wartością w ludziach jest dla niej liczba zer na koncie. Traktuje wszystkich
z wyższością i nieustannie szuka uwagi, lecz w mig spławia tych, którzy nie obsypią jej złotem i diamentami. Nie ma za
dużo w głowie, toteż używa prostego języka. Ma jednak świadomość, że to tylko wygląd się liczy, a tutaj świeci niczym
najjaśniejsza z gwiazd.

\section{Wygląd}
\begin{itemize}
	\item Tandetny, wyzywający strój
	\item Strzaskana twarz od solary
	\item Tony makijażu
	\item Dekolt\footnote{DUŻY.}
\end{itemize}

\section{Wyposażenie}
\begin{itemize}
	\item Minimalistyczna, dżinsowa spódniczka
	\item Białe, wysokie kozaczki lub świecące szpilki
	\item Kolczyk w pępku\footnote{Ale żebys się do niego nie przyzwyczaiła...}
	\item Guma do żucia\footnote{W buzi, ale patrz zadanie za 10pkt.}
\end{itemize}


\section{Zadania}
1pkt:
\begin{itemize}
	\item Wyciągnąć znajomą na solarkę
	\item Zaprzyjaźnić się z inna blacharą
	\item Wyszydzić czyjąś inteligencję lub stan materialny\footnote{Tak, żeby uwierzył.}
	\item Dostać drinka w barze od nowo poznanego mężczyzny 
\end{itemize}
\newpage

2pkt:
\begin{itemize}
	\item Zostać zaproszonym na imprezę przez nieznanego faceta\footnote{Znajomi znajomych się nie liczą}
	\item Pójść na dyskotekę i sprowokować kogoś, do zrobienia czegoś głupiego w zamian za taniec
	\item Sprawić, że nieznajomy zagwiżdże na Twój widok na ulicy\footnote{Bez mówienia mu o tym}
	\item Sprawić, że starsza osoba skomentuje na głos 'dzisiejszą młodzież'
\end{itemize}

5pkt:
\begin{itemize}
	\item Przejechać się wozem za 200+ koła\footnote{Nie, autobus się nie liczy.}
	\item Doprowadzić obcego do tego, aby Cię obraził, po czym dać mu solidnego plaskacza
\end{itemize}

10pkt:
\begin{itemize}
	\item Sprawić, żeby ktoś poprosił Cię o zrobienie loda. Zgodzić się i rozmyślić\footnote{Opcjonalne.} 
		w ostatnim\footnote{Ostatnim!} momencie
\end{itemize}
\end{document}


%% end of file `template.tex'.
